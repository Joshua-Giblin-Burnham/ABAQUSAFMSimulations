Atomic Force Microscopy (AFM) is a versatile three-dimensional topographic technique implementing a mechanical probe to raster-scan and image sample surfaces. The technique provides reliable nanometer measurements of materials \cite{nagashima1996nanoscopic, kempf1998nanohardness,goken1999microstructural,yamamoto2000afm} and has become a valuable tool with a diverse range of applications in areas such as materials physics, nanotechnology, electronics, and biology\cite{JALILI2004907,SANTOS2004133, goken1999microstructural}. In addition, AFM can image under natural conditions, such as in aqueous solutions and in real-time, allowing imaging of cell dynamics and biological processes. This includes imaging protein unfolding \cite{hughes2016physics} and conformational changes\cite{moody2006atomic}, alongside, characterising microbial surfaces\cite{wright2006application,dufrene2004using}. The ability to image and measure the physical properties of microbial surfaces can give important insight into microbiology, such as improving inhibition and cellular damage produced by antimicrobial compounds\cite{wright2006application, TYAGI2010797}. 

However, there are limited computational recreations of AFM imaging, and the area could benefit from greater tools to aid in interpreting surface characteristics. As with any experimental technique, AFM has limitations. Various effects can lead to ambiguity in images and image artefacts. A key source of error is a consequence of the resolution being directly dependent on imaging force and the probe geometry \cite{dufrene2002atomic}. Imaging with large forces can dramatically reduce image resolution and damage the surface. Furthermore, probe geometry and its interaction with the sample is important to image contrast \cite{dufrene2002atomic}. An AFM image is a convolution of the probe geometry and the sample's topology. Therefore, the tip-sample convolution produces a trace of the tip geometry over the surface, broadening protrusions and narrowing holes in the surface.Furthermore, other errors may arise due to environmental surroundings. For example, environmental vibrations can cause the probe to vibrate and produce artefacts and blur. Similarly, thermal drift is produced from prolonged usage, which causes the probe to expand/ contract thermally and produce deviations in the system.

Currently, simulations of AFM imaging use a hard-sphere model and neglect tip indentation. Recent work by Amyot R, Flechsig H  \textit{et al.}. \cite{amyot2020bioafmviewer} produced the BiomolecularAFMviewer which, similar to this project, uses protein structural data to simulate AFM images. This has proven the utility in interpreting experimental observations and reconstruct resolution-limited experimental images\cite{amyot2022simulation}. However, the simulations do not account for indentation into the surface, surface deflection, off-axial forces that produce sliding and friction, or elastic properties of the surface. This approach could be improved by accounting for force curves and the indentation of the tip with elastic properties of the material. 

Other computational simulations of AFM have been used to produce such force curves that reflect these properties. However, these applications study quantitative AFM results as opposed to imaging \cite{liu2019finite,han2021modified,roduit2009stiffness,kontomaris2020hertz,senda2016computational}. Previous work has shown the viability of the commercial software ABAQUS and Finite Element Modelling (FEM) in the study of indention in AFM; Liu \textit{et al.}\cite{liu2019finite} validated a FEM model with less than 10\% error when comparing the simulated force-indentation curves with the experimental data. The work by Rajabifar \textit{et al.}.\cite{rajabifar2021fast} simulated the viscoelasticity contact between an AFM tip and a surface showing a fast and accurate use of FEM to simulate AFM indentation and the associated force curves.  

Our research presents novel FEM and computational methods to model and explore AFM images, specifically in biomolecular samples. The primary work of this research focuses on improving the modelling of previous AFM simulations from a hard sphere model to a model with tip surface interactions. FEM simulations enable us to simulate tip indentation and generate force curves that incorporate more intricate forces and dynamics. We employ several FEM simulations to investigate contact models of indentation and sample compression in AFM imaging. The overall goal was to produce more accurate images and artefacts. Our implementation of FEM utilises the commercial engineering software ABAQUS. 