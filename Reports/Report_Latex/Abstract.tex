\section*{Abstract}

Atomic Force Microscopy (AFM) is a versatile three-dimensional topographic technique implementing a mechanical probe to raster-scan and image sample surfaces. The technique provides reliable nanometer measurements of materials \cite{nagashima1996nanoscopic, kempf1998nanohardness,goken1999microstructural,yamamoto2000afm} and has become a valuable tool with a diverse range of applications in areas such as materials physics, nanotechnology, electronics, and biology\cite{JALILI2004907,SANTOS2004133, goken1999microstructural}. However,   there are limited computational recreations of AFM imaging, and the area could benefit from greater tools to aid in interpreting surface characteristics. Consequently, this research presents novel computational modelling of AFM imaging, specifically for biomolecular samples. Currently, simulations of AFM imaging use a hard-sphere model and neglect tip indentation\cite{amyot2020bioafmviewer}. This research implemented Finite Element Modelling (FEM) and the commercial software ABAQUS to simulate AFM tip indentation and produce force curves that accounted for contact dynamics. Initial tests verified the accuracy of ABAQUS, focusing on the indentation of elastic half-spaces and spheres. The elastic half-space simulations showed good agreement with the theoretical models. In contrast, the results indicate that simple Hertzian models underestimate the elastic modulus of spherical samples and instead require Double Contact models. Moreover, a novel formulation of the Double Contact model for conical indenters demonstrated significant predictive power over a range of surface radii. Next, a FEM approach was applied to analyse the compression of simple hemispheres and periodic surfaces during AFM imaging. These simulations highlighted the dependency of the elastic behaviour on the contact radius and tip convolution. Our results indicated that larger indenters require larger forces to compress the sample to the same extent. In addition, Fourier analysis of the simulated AFM contours elucidated a possible novel trend, that larger indentation forces recover more of a surface's periodicity. Finally, a FEM simulation of AFM imaging was applied to simulate the appearance of B-DNA Dodecamer. Simulations used an assembly of the AFM tip and the biomolecule surface to produce individual indentations across the sample. Subsequently, contours of constant force are used to return an AFM image. These simulations show the viability of the FEM approach in reproducing the AFM dynamics and provide a wealth of extensions to be explored.